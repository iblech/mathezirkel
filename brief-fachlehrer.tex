\documentclass{zirkelbrief}

\geometry{tmargin=1cm,bmargin=1cm,lmargin=2.5cm,rmargin=2.5cm}

\begin{document}

\renewcommand{\anschrift}{%
      Holbein-Gymnasium Augsburg \\
      Fachbereich Mathematik \\
      Hallstraße 10 \\
      86150 Augsburg}
\renewcommand{\datum}{2.9.2013}
\renewcommand{\betreff}{Matheschülerzirkel der Universität Augsburg}

\makeletterhead

Sehr geehrte Lehrkraft der Mathematik,

beginnend mit dem Schuljahr 2013/2014 bietet die Universität Augsburg
einen \emph{Mathe\-matik\-zir\-kel} für interessierte Schülerinnen und Schüler der
Jahrgangsstufen~5 bis~12 an.

Dabei sollen bei regelmäßig stattfindenden Treffen, \emph{Präsenzzirkeln}, in
den Räumlichkeiten der Universität spannende mathematische Themen den Schülern
nähergebracht werden; es geht nicht um Nachhilfe oder Vorwegnahme von
Unterrichtsinhalten, sondern um Beschäftigung mit orthogonalen Themen wie etwa
Spieltheorie, Zahlenrätsel, Fraktale oder Graphentheorie. Falls es gewünscht
wird, ist auch eine Vorbereitung auf Wettbewerbe wie den Landes- und
Bundeswettbewerb und dem Känguru der Mathematik möglich.

Für diejenigen Schüler, die an den Präsenzzirkeln nicht teilnehmen können, gibt
es schriftliche \emph{Korrespondenzzirkel}. Dabei erhalten die Schüler von uns
Materialien und bearbeiten Übungsblätter, welche wir dann korrigieren.

Die Kurse werden von Mitarbeiterinnen und Mitarbeitern des
Instituts für Mathematik gehalten, die zum Teil schon selber von derartigen
Programmen profitierten. Die Kosten übernimmt die Universität, sodass die Teilnahme für die Schüler
kostenlos ist. Die Präsenzzirkel finden außerhalb der Schulzeiten
statt.

Für alle Schüler, die sich für eine Teilnahme an einem Mathematikzirkel
interessieren, findet eine
\begin{quote}
    Eröffnungsveranstaltung am ??.11.2013 um ??:?? Uhr mit einem Vortrag von \\
    Prof. Dr. Jost-Hinrich Eschenburg, \emph{Das
    Geheimnis der Zahl 5}
\end{quote}
im Hörsaal 1004/T des Physik-Hörsaalzentrums der Universität statt, eine
Anmeldung ist nicht erforderlich. Die konkreten Termine für die
ersten Präsenzzirkel und der Ablauf der Korrespondenzzirkel werden auch dort
abgesprochen.

Wir bitten Sie, Ihre Schüler auf unser Angebot aufmerksam zu machen und sie zur
Teilnahme zu ermuntern. \emph{Einzige Teilnahmevoraussetzung ist Spaß und
Interesse an der Mathematik.} Für Rückfragen stehen wir Ihnen jederzeit
telefonisch und schriftlich zur Verfügung. Wenn gewünscht, kommen wir auch
gerne zu Ihnen an die Schule, um Ihnen und Ihren Kolleginnen und Kollegen oder
Ihren Schülern das Angebot persönlich zu erläutern. Vielen Dank für Ihre
Unterstützung.

Mit freundlichen Grüßen,

\end{document}

Zu diskutieren:
* doch "profitiert haben"?
* Kathrin: Ausserdem: Die Praesenzzirkel sind fuer alle, oder?
  Ingo: Was meinst du damit?
* Wie von Kathrin beobachtet, passt der Satz "Die Präsenzzirkel werden
  natürlich außerhalb der Schulzeiten stattfinden." nicht gut in den Fluss.
* Alternative: "Wettbewerbe wie Känguru der Mathematik, Landers- oder
  Bundeswettbeweb."
