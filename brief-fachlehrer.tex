\documentclass{zirkelbrief}

\geometry{tmargin=0.5cm,bmargin=1.5cm,lmargin=1.8cm,rmargin=2.7cm}

\begin{document}

\renewcommand{\anschrift}{%
      Holbein-Gymnasium Augsburg \\
      Fachbereich Mathematik \\
      Hallstraße 10 \\
      86150 Augsburg}
\renewcommand{\datum}{\today}
\renewcommand{\betreff}{Matheschülerzirkel der Universität Augsburg}


\makeletterhead

Sehr geehrte Lehrerin, sehr geehrter Lehrer,

beginnend mit dem Schuljahr 2013/2014 bietet die Universität Augsburg
einen \emph{Mathe\-matik\-zir\-kel} für interessierte Schülerinnen und Schüler der
Jahrgangsstufen~5 bis~12 an.

Dabei sollen bei regelmäßig stattfindenden Treffen, \emph{Präsenzzirkeln}, in
den Räumlichkeiten der Universität den Schülern spannende mathematische Themen
nähergebracht werden; es geht nicht um Nachhilfe oder Vorwegnahme von
Unterrichtsinhalten, sondern um Beschäftigung mit anderen Themen wie etwa
Spieltheorie, Zahlenrätseln, Fraktalen oder Graphentheorie. Falls es gewünscht
wird, ist auch eine Vorbereitung auf Wettbewerbe wie den Landes- und
Bundeswettbewerb oder den Känguru der Mathematik möglich.

%Für diejenigen Schüler, die an den Präsenzzirkeln nicht teilnehmen können, gibt
%es schriftliche \emph{Korrespondenzzirkel}.
Zusätzlich bieten wir schriftliche \emph{Korrespondenzzirkel} an, an denen auch diejenigen Schüler teilnehmen können, die zu den Präsenzzirkeln verhindert sind. Dabei erhalten die Schüler von uns
Materialien und bearbeiten Übungsblätter, welche wir dann korrigiert zurückgeben.

Die Kurse werden von Mitarbeiterinnen und Mitarbeitern des Instituts für
Mathematik gehalten, die zum Teil schon selbst von derartigen Programmen
profitiert haben. Die Kosten übernimmt die Universität Augsburg, sodass die
Teilnahme für die Schüler kostenlos ist. Natürlich finden die Präsenzzirkel
außerhalb der Unterrichtszeiten statt.

Für alle Schüler, die sich für eine Teilnahme an einem Mathematikzirkel
interessieren, findet eine Eröffnungsveranstaltung am 9.11.2013 um 10:00 Uhr mit einem Vortrag von
\linebreak
\vspace{-2.2em}
\begin{center}
    Prof. Dr. Jost-Hinrich Eschenburg\\
    \emph{Was sind eigentlich die Zahlen?}
\end{center}
\vspace{-1em}
im Hörsaal 1004 des Physikhörsaalzentrums der Universität statt, eine
Anmeldung ist nicht erforderlich. Die konkreten Termine für die
ersten Präsenzzirkel und der Ablauf der Korrespondenzzirkel werden auch dort
abgesprochen.

Ich bitte Sie, Ihre Schüler auf unser Angebot aufmerksam zu machen, sie zur
Teilnahme zu ermuntern und geeignete Schüler gezielt
anzusprechen. \emph{Einzige Teilnahmevoraussetzung ist Spaß und
Interesse an der Mathematik.} Selbstverständlich können wir Ihnen weitere Flyer
zukommen lassen. Für Rückfragen stehe ich Ihnen jederzeit zur Verfügung. Wenn gewünscht, kommen wir auch
gerne an Ihre Schule, um Ihren Schülern eine Kostprobe zu geben. Vielen Dank für Ihre
Unterstützung.

Mit freundlichen Grüßen,

\vspace{1cm}

Prof. Dr. Marco Hien

\vspace{-0.2cm}

{\small Studiendekan der Mathematik}
\end{document}
