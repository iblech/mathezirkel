\documentclass{zirkelbrief}

\begin{document}

\renewcommand{\anschrift}{%
      Holbein-Gymnasium Augsburg \\
      Fachbereich Mathematik \\
      Hallstraße 10 \\
      86150 Augsburg}
\renewcommand{\datum}{2.9.2013}
\renewcommand{\betreff}{Matheschülerzirkel der Universität Augsburg}
\renewcommand{\absender}{%
      \textbf{Dokto Rand} \\
      \ \\
      Lehrstuhl für spannende Mathematik \\
      Universitätsstr. 14 \\
      86159 Augsburg \\
      \ \\
      Telefon \> +49 (0) 821 598 -- ???? \\
      Telefax \> +49 (0) 821 598 -- ???? \\
      \textsf{dokto.rand@math.uni-augsburg.de} \\
      \textsf{http:/\!/www.math.uni-augsburg.de/alg} \\}

\makeletterhead

Sehr geehrte Frau Mathespaß,

ich bin ein Doktorand in Mathematik an der Universität Augsburg und schreibe
Ihnen, weil ein ehemaliger Schüler von Ihnen, Anne Hatspaßanmathe, Sie mir
gegenüber als besonders engagierten Mathematiklehrer an Ihrer Schuler
beschrieben hat.

Hintergrund ist folgender: Die Universität Augsburg bietet ab dem dem Schuljahr 2013/2014 einen
Mathematikzirkel für interessierte Schülerinnen und Schüler der
Jahrgangsstufen~5 bis~12 an. Dabei wird es in den Räumlichkeiten der
Universität Augsburg regelmäßig stattfindende Treffen geben, bei denen wir
spannende Themen der Mathematik abseits des Schulunterrichts behandeln wollen;
[da Ihrstadt ja recht weit von Augsburg entfernt liegt,] bieten wir außerdem
schriftliche Korrespondenzzirkel an.

Diesbezüglich verschicken wir in den nächsten Tagen an alle Gymnasien Schwabens
entsprechende Informationsmaterialien. Wenn Sie sich vorab darüber informieren
möchten, können Sie das im Internet auf
\textsf{http:/\!/www.math.uni-augsburg.de/...} tun. Von Projekten dieser Art an
anderen Orten haben wir erfahren, dass solche Briefe in der Menge der
Korrespondenz manchmal untergehen. Daher würde ich mich sehr freuen, wenn Sie
Sorge tragen könnten, dass unser Anliegen an Ihrer Schule an die Schüler
weitergetragen wird.

Vielen Dank für Ihre Unterstützung, mit freundlichen Grüßen,

\ \\
\ \\

\qquad Dokto Rand

\qquad (stellvertretend für das gesamte Organisationsteam)

\end{document}
