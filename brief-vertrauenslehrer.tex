\documentclass{zirkelbrief}

\begin{document}

\renewcommand{\anschrift}{%
      Holbein-Gymnasium Augsburg \\
      Fachbereich Mathematik \\
      Hallstraße 10 \\
      86150 Augsburg}
\renewcommand{\datum}{\today}
\renewcommand{\betreff}{Matheschülerzirkel der Universität Augsburg}
\renewcommand{\absender}{%
      \textbf{Ingo Blechschmidt} \\
      \ \\
      Lehrstuhl für Algebra und Zahlentheorie \\
      Universitätsstr. 14 \\
      86159 Augsburg \\
      \ \\
      Telefon \> +49 (0) 821 598 -- 5601 \\
      Telefax \> +49 (0) 821 598 -- 2090 \\
      \textsf{blechschmidt@math.uni-augsburg.de} \\}

\makeletterhead

Sehr geehrte Frau Mathespaß,

ich bin ein Doktorand in Mathematik an der Universität Augsburg und schreibe
Ihnen, weil ich Sie auf ein neu initiiertes Projekt der Universität aufmerksam machen
möchte: Ab dem Schuljahr~2013/2014 bieten wir einen Mathematikzirkel für
interessierte Schülerinnen und Schüler der Jahrgangsstufen~5 bis~12 an. Dabei
wird es in den Räumlichkeiten der Universität Augsburg regelmäßig stattfindende
Treffen geben, bei denen wir spannende Themen der Mathematik abseits des
Schulunterrichts behandeln wollen; [da Ihrstadt ja recht weit von Augsburg
entfernt liegt,] bieten wir außerdem schriftliche Korrespondenzzirkel an, bei
denen die Schüler von uns Materialien erhalten und Übungsblätter bearbeiten,
welche wir dann korrigiert zurückgeben.

Diesbezüglich verschicken wir in den nächsten Tagen an alle Gymnasien Schwabens
und ein paar weitere Schulen entsprechende Informationsmaterialien. Nun ist es
so, dass wir von Projekten dieser Art an anderen Orten erfahren haben, dass
solche Briefe in der Menge der Korrespondenz leider manchmal untergehen.
Deshalb erhalten für das XYZ-Gymnasium speziell Sie diesen Brief vorab:

Wir haben unsere Studenten um Mithilfe gebeten und sie gefragt, ob sie an ihrer
ehemaligen Schule jeweils eine besonders engagierte Mathematik-Lehrerin oder einen besonders
engagierten Mathematik-Lehrer kennen, die bzw. der
vielleicht unser Projekt unterstützen würde. Für das XYZ-Gymnasium hat sich
Ihre ehemalige Schülerin Anne Hatspaßanmathe gemeldet: Anne hat Sie in bester
Erinnerung und dachte, dass Sie die beste Kandidatin wären, um an Ihrer Schule
dafür Sorge zu tragen, dass unser Anliegen an die Schüler
weitergegeben wird. Darum würde ich Sie also ganz herzlich bitten.

Außerdem hoffe ich, dass Sie mir diese Verletzung des Dienstwegs verzeihen, und
dass sich kein anderer Lehrer unberücksichtigt fühlt. Unsere Studentenerhebung
hatte notgedrungen großen Stichprobencharakter.

Vielen Dank für Ihre Unterstützung!

Mit freundlichen Grüßen

\ \\
\ \\

Ingo Blechschmidt

\vspace{-0.2cm}

{\small (stellvertretend für das gesamte Organisationsteam)}

\end{document}
