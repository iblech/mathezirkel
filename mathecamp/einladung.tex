\documentclass[12pt]{zettel}

%\renewcommand{\gregor}{\put(10.0,-3.5){\includegraphics[scale=0.18]{campgregor}}}

\usepackage{framed}
\definecolor{shadecolor}{rgb}{.97,.97,.97}

\geometry{tmargin=1.5cm,bmargin=1.5cm,lmargin=2.5cm,rmargin=2.5cm}

\renewcommand{\gregor}{\put(13.2,-3.0){\includegraphics[scale=0.18]{cover}}}
\begin{document}

\renewcommand{\betreff}{Mathecamp des Matheschülerzirkels Augsburg vom 16. bis
20. August}

\makeletterhead{}
\begin{picture}(0,0)
  \put(7.0,-19.0){%
    \includegraphics[scale=0.2]{campgregor}
  }
\end{picture}
\vspace{-2em}

Liebe Schülerinnen und Schüler, liebe Eltern,

wir laden euch herzlich zum ersten Mathecamp des Matheschülerzirkels Augsburg
ein. Dort werden wir mit euch tolleXXXspannendeXXXvielfältigeXXXinteressante mathematische Themen verstehen und die
Freizeit in den Ferien genießen. Darüber hinaus ist das Camp eine gute
Möglichkeit, Gleichgesinnte kennenzulernen und neue Freunde zu finden.

\begin{tabbing}
  \hspace{2.2cm} \= \kill
  \textbf{Was?} \> Mathematik und Spaß in den Ferien \\[0.3em]
  \textbf{Wann?} \> 16. bis 20. August 2014 (Samstag bis Mittwoch) \\[0.3em]
  \textbf{Wo?} \> Bruder-Klaus-Heim in Violau, Schullandheim der Diözese
  Augsburg \\[0.3em]
  \textbf{Für wen?} \> \begin{minipage}[t]{\dimexpr\textwidth-2.3cm}
  Einzige Teilnahmevoraussetzung ist Spaß und Interesse an
  Mathematik.
  Jeder kann mitkommen, auch, wenn man nicht bei den Zirkeln
  mitgemacht hat.\end{minipage} \\[0.3em]
  \textbf{Kosten?} \> 70,-- \texteuro
\end{tabbing}

An jedem Tag werden zwei Arbeitsgruppen stattfinden. Dabei behandeln wir spannende
Bereiche der Mathematik, die wir in den Präsenz- und Korrespondenzzirkeln noch nicht gesehen haben.
Selbstverständlich gibt es für verschiedene Alters- und Vorkenntnisstufen angepasste Kurse
mit unterschiedlichen Themen und Schwierigkeiten, sodass immer für jeden
etwas dabei ist.

Daneben stehen Spiele- und Bastelstunden sowie die Nutzung der
Sternwarte des Heims auf dem Programm. Natürlich hoffen wir auf
schönes Wetter, damit wir auch die Tiere auf dem Gelände besuchen
und am Abend am Lagerfeuer grillen und Musik machen können.
Weiterhin stehen uns eine Pizzabäckerei, ein Volleyballplatz und ein
Fußballplatz zur Verfügung, sodass es sicher niemandem langweilig
wird. Außerdem wird es zwei Vorträge von
einer auswärtigen Mathematikerin und einem Mathematiker geben.

\vspace{\medskipamount}

\begin{minipage}{0.6\textwidth}
Wenn ihr teilnehmen möchtet, bittet eure Eltern das beiliegende Anmeldeformular
bis zum 12. Juli 2014 auszufüllen und an uns zurückzuschicken.
\end{minipage}

\vspace{\medskipamount}

\begin{minipage}{0.4\textwidth}
Wenn ihr Freunde
habt, die Spaß an Mathe haben und ebenfalls gerne mit auf das
Camp fahren möchten, aber nicht an den Zirkeln beteiligt waren, so
können diese trotzdem gerne mitkommen.
\end{minipage}

\newpage

Getragen wird das Camp vom Mathematisch-Physikalischen Verein e.\,V. Das
Betreuer- und Organisationsteam um Sven Prüfer und Kathrin Helmsauer besteht
aus Doktoranden und Mitarbeitern des Instituts, die in ihrer Freizeit
ehrenamtlich interessierten Schülerinnen und Schülern Mathematik näherbringen
wollen. Sowohl Sven und Kathrin als auch mehrere weitere Betreuer haben
bereits Erfahrungen in der Jugendarbeit und bei der Durch\-füh\-rung von
Sommercamps.

Wir bitten, den Betrag bis zum 1. August 2014 auf unser Konto zu überweisen.
In ihm sind die Kosten für An- und Abreise, Unterkunft, Verpflegung
(Vollpension) und Freizeitaktivitäten enthalten; wir Betreuer arbeiten
ehrenamtlich. Familien, die sich den Eigenbeteiligung nicht leisten können, bitten
wir, mit uns informal Kontakt aufzunehmen. Im Rahmen der finanziellen Möglichkeiten des
Vereins können wir gegebenenfalls auf den Beitrag verzichten.

\vspace{-0.7em}
\begin{tabbing}
  \qquad\quad \= Verwendungszweck:\, \= \kill
  \> Kontoinhaber: \> Matheschülerzirkel Augsburg \\
  \> Konto-Nr.: \> 12345678 \\
  \> BLZ: \> 12345678 \\
  \> IBAN: \> 2437243784237878342 \\
  \> BIC: \> 23487432789432 \\
  \> Betrag: \> 70,-- \texteuro{} bzw. 75,-- \texteuro, falls Bettwäsche
  gestellt werden soll \\
  \> Verwendungszweck: \> Mathecamp \emph{Vorname Nachname}
\end{tabbing}
\vspace{-0.7em}

Das Camp beginnt am 16. August zwischen 10:00 Uhr und 11:00 Uhr. Ihr könnt
entweder individuell zu dieser Zeit anreisen oder euch um 09:30 Uhr auf dem Campus der
Universität einfinden, um dann gemeinsam mit uns im Reisebus nach Violau zu fahren.
Die Abreise ist am 20. August ab 16:30 Uhr. Ihr könnt euch entweder von euren
Eltern abholen lassen oder mit uns zurück nach Augsburg fahren. Dort kommen wir
gegen 17:30 Uhr an der Universität an.

Wir schicken euch Ende Juli eine Übersicht mit
letzten Details und Kontaktdaten vor Ort. Bei Fragen stehen wir euch jederzeit
zur Verfügung. Bitte zögert nicht, uns dazu telefonisch unter 0821/598-5805
(Sven Prüfer) oder per Mail an \textsf{mathezirkel@math.uni-augsburg.de} zu kontaktieren.

Wir hoffen, dass wir
euch auf dem Mathecamp begrüßen dürfen und dass
ihr euch darauf genauso freut wie wir!

\vspace{2em}

Euer Team vom Mathezirkel

{\small Meru Alagalingam, Tim Baumann, Martin Baur, Ingo Blechschmidt, Philipp Düren,
Alexander Engel, Johanna Fleckenstein, Kathrin Helmsauer, Prof. Dr. Marco Hien,
Christian Hübschmann, Simon Kapfer, Sven Prüfer, PD Dr. Peter Quast,
Lisa Reischmann, Peter Uebele, Prof. Dr. Timo Schürg, Carina Willbold,
Christopher Wulff, Stephanie Zapf}

\vfill

PS: Pro Kind entstehen uns Kosten von etwa~200~\texteuro. Die niedrige
Eigenbeteiligung ist nur durch eine großzügige und einmalige finanzielle
Unterstützung des Instituts für Mathematik möglich, das die restlichen
130~\texteuro{} pro Teilnehmer übernimmt. Falls Sie unsere Arbeit unterstützen
und dadurch auch zukünftig Veranstaltungen dieser Art ermöglichen möchten,
können Sie gerne einen höheren Beitrag überweisen. Als gemeinnütziger Verein
stellen wir dann eine Spendenbescheinigung über den zusätzlichen
Betrag aus.

\end{document}
