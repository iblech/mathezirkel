\documentclass{../vorlagen/werbung/lehrerbriefe/zirkelbrief}

\geometry{tmargin=3.5cm,bmargin=3.5cm,lmargin=3.5cm,rmargin=3.5cm}

\begin{document}

\renewcommand{\anschrift}{%
      Herr Lepperdinger \\
      Hallstraße 10 \\
      86150 Augsburg}
\renewcommand{\datum}{\today}
\renewcommand{\betreff}{Matheschülerzirkel der Universität Augsburg}
\renewcommand{\absender}{%
      \textbf{Ingo Blechschmidt} \\
      \ \\
      Lehrstuhl für Algebra und Zahlentheorie \\
      Universitätsstr. 14 \\
      86159 Augsburg \\
      \ \\
      Telefon \> +49 (0) 821 598 -- 5601 \\
      Telefax \> +49 (0) 821 598 -- 2090 \\
      \textsf{blechschmidt@math.uni-augsburg.de} \\}

\makeletterhead

Sehr geehrter Herr Lepperdinger,

mein Name ist Ingo Blechschmidt, ich bin wissenschaftlicher Mitarbeiter
am mathematischen Institut der Universität Augsburg.

Seit bald zwei Jahren organisiert ein Team von Doktorandinnen und
Doktoranden des Instituts ehrenamtlich in ihrer Freizeit den Augsburger
Matheschülerzirkel. In dessen Rahmen können an Mathematik begeisterte
Schü\-ler\-in\-nen und Schüler aller weiterführenden Schulen alle zwei Wochen
zu uns auf den Campus kommen, um in kleinen Gruppen spannende Mathematik
abseits des Schulunterrichts zu betreiben. Für weiter entfernt wohnende
Schülerinnen und Schüler bieten wir schriftliche Korrespondenz per Post.

Es geht uns nicht um Nachhilfe, sondern um Förderung von an Mathematik
interessierten Schülerinnen und Schülern. Wir zeigen den Kindern und
Jugendlichen, wie vielseitig Mathematik sein kann, und führen
Gleichgesinnte zusammen. Wir schließen also dieselbe Art Lücke, wie sie
Schachclubs oder Fußballvereine in ihren Bereichen schließen.

Der Matheschülerzirkel ist in Schwaben einzigartig. Bisher nahmen etwa 300
Kinder und Jugendliche unser Angebot wahr. Unsere letzte
Auftaktveranstaltung wurde von Herrn Köhler begleitet.

Ich schreibe Sie in der Hoffnung an, dass Sie uns in einer Frage um
finanzielle Förderung weiterhelfen können. Im August werden wir im
Bruder-Klaus-Heim in Violau zum zweiten Mal ein einwöchiges Mathecamp
veranstalten. Dabei bieten wir sowohl ein umfangreiches fachliches
Programm als auch diverse Freizeitaktivitäten an; letztes Jahr hatten
sich dazu mehr als 100 Kinder und Jugendliche angemeldet.

Wir suchen finanzielle Unterstützung, um auf unbürokratische Art und
Weise auch Kinder aus materiell schlechter gestellten Familien mitnehmen
zu können, die sich den Eigenbeitrag von etwa 250 EUR (der sich
hauptsächlich aus den Unterbringungskosten zusammensetzt) nicht leisten
können.

Außerdem würden wir uns wünschen, den Eigenbeitrag für alle
Teilnehmenden etwas reduzieren zu können -- wir glauben, dass der Preis
sonst eine Hemmschwelle schafft, die insbesondere solche begabten Kinder
von einer Teilnahme abhält, die zwar viel Freude an der Mathematik
empfinden, deren Talent bisher aber nicht spezifisch gefördert wird.

Wir sind Ihnen für jede Hilfe beim Einwerben finanzieller Förderung sehr
dankbar. Telefonisch können Sie mich unter 0821/598-5601 erreichen. Die
Homepage des Matheschülerzirkels finden Sie unter
\textsf{http:/\!/www.math.uni-augsburg.de/schue\-ler/mathezirkel/}.

Mit freundlichen Grüßen

\ \\
\ \\

Ingo Blechschmidt

\end{document}
