\documentclass{zirkelbrief}

\usepackage{geometry}
\geometry{tmargin=2cm,bmargin=2cm,lmargin=2.5cm,rmargin=2.5cm}

\begin{document}

\renewcommand{\anschrift}{%
      Dieter Götzl (Ref. V.7) \\
      Bayerisches Staatsministerium für \\
Bildung und Kultus, Wissenschaft und Kunst \\
      80327 München}
\renewcommand{\datum}{\today}
\renewcommand{\betreff}{Matheschülerzirkel der Universität Augsburg}
\renewcommand{\absender}{%
      \textbf{Ingo Blechschmidt} \\
      \ \\
      Matheschülerzirkel \\
      Institut für Mathematik \\
      Universitätsstr. 14 \\
      86159 Augsburg \\
      \ \\
      Telefon \> +49 (0) 821 598 -- 5601 \\
      Telefax \> +49 (0) 821 598 -- 2090 \\
      \textsf{blechschmidt@math.uni-augsburg.de} \\}

\makeletterhead

Sehr geehrter Herr Götzl,

mein Name ist Ingo Blechschmidt, ich bin wissenschaftlicher Mitarbeiter
am mathematischen Institut der Universität Augsburg und mir wurde von Frau Susanne Täufer empfohlen, mich an Sie zu wenden.

Ich bin Teil des Organisationsteams des Augsburger Matheschülerzirkels, welcher mathematikbegeisterten Jugendlichen aus dem Großraum Augsburg die Möglichkeit bietet, spannende Mathematik abseits des Schulunterrichts zu betreiben. Dazu gibt es regelmäßige Angebote während des Schuljahres, die sowohl an der Universität als auch per Post wahrgenommen werden können. Außerdem organisieren wir eine regionale Landesrunde der Deutschen Mathematik-Olympiade sowie ein mehrtägiges Mathecamp in den Sommerferien.

Es geht uns nicht um Nachhilfe, sondern um Förderung von an Mathematik
interessierten Schülerinnen und Schülern. Wir zeigen den Kindern und
Jugendlichen, wie vielseitig Mathematik sein kann, und führen
Gleichgesinnte zusammen. Wir schließen also dieselbe Art Lücke, wie sie
Schachclubs oder Fußballvereine in ihren Bereichen schließen.

Weitere Informationen zu unseren Veranstaltungen können Sie der Anlage entnehmen.


% Seit bald zwei Jahren organisiert ein Team von Doktorandinnen und
% Doktoranden des Instituts ehrenamtlich in ihrer Freizeit den Augsburger
% Matheschülerzirkel. In dessen Rahmen können an Mathematik begeisterte
% Schü\-ler\-in\-nen und Schüler aller weiterführenden Schulen alle zwei Wochen
% zu uns auf den Campus kommen, um in kleinen Gruppen spannende Mathematik
% abseits des Schulunterrichts zu betreiben. Für weiter entfernt wohnende
% Schülerinnen und Schüler bieten wir schriftliche Korrespondenz per Post.
% 
% Es geht uns nicht um Nachhilfe, sondern um Förderung von an Mathematik
% interessierten Schülerinnen und Schülern. Wir zeigen den Kindern und
% Jugendlichen, wie vielseitig Mathematik sein kann, und führen
% Gleichgesinnte zusammen. Wir schließen also dieselbe Art Lücke, wie sie
% Schachclubs oder Fußballvereine in ihren Bereichen schließen.
% 
% Der Matheschülerzirkel ist in Schwaben einzigartig. Bisher nahmen etwa 300
% Kinder und Jugendliche unser Angebot wahr. Unsere letzte
% Auftaktveranstaltung wurde von Herrn Köhler begleitet.

% Ich schreibe Sie in der Hoffnung an, dass Sie uns in einer Frage zu
% finanzieller Förderung weiterhelfen können. Im August werden wir im
% Bruder-Klaus-Heim in Violau zum zweiten Mal ein einwöchiges Mathecamp
% veranstalten. Dabei bieten wir sowohl ein umfangreiches fachliches
% Programm als auch diverse Freizeitaktivitäten an; letztes Jahr hatten
% sich dazu mehr als 100 Kinder und Jugendliche angemeldet.

% \newpage

Unsere laufenden Kosten für Materialien, die Matheolympiade und Auftakt- sowie Abschlussveranstaltung konnten in den letzten beiden Jahren durch Spenden getragen werden. Da wir ehrenamtlich arbeiten, können wir also bis auf das Mathecamp alle unsere Veranstaltungen für die Schülerinnen und Schüler kostenlos durchführen. Für das Mathecamp aber entstehen uns pro Teilnehmenden Kosten in Höhe von 230 Euro, welche wir vollständig auf die Teilnehmenden umlegen müssen. Wir haben die Sorge, dass insbesondere materiell schlechter gestellte Familien nicht in der Lage sind, diesen Betrag aufzubringen und wir versuchen daher, ihnen beim Eigenbeitrag entgegenzukommen. Zusätzlich würden wir gerne den Teilnahmebeitrag für alle Familien verringern, um so die Hemmschwelle vor einer Teilnahme herabzusetzen. Daher sind wir auch beim Mathecamp auf Spenden angewiesen.

% dadurch von einer Teilnahme abgehalten werden. Daher sind wir auf der Suche nach finanzieller Unterstützung. Wir möchten den Eigenbeitrag von Jugendlichen aus materiell schlechter gestellten Familien deutlich reduzieren und zusätzlich den Teilnahmebeitrag für alle Familien verringern, um so die Hemmschwelle vor einer Teilnahme herabzusetzen.


% Dennoch entstehen uns Kosten für Materialien sowie die Unterkunft für das Mathecamp, welches wir zum Selbstkostenpreis anbieten. Wir sind auf der Suche nach finanzieller Unterstützung, insbesondere um möglichst vielen Jugendlichen die Teilnahme am Mathecamp zu ermöglichen.

% um auf unbürokratische Art und
% Weise auch Kinder aus materiell schlechter gestellten Familien mitnehmen
% zu können, die sich den Eigenbeitrag von etwa 230 EUR (der sich
% hauptsächlich aus den Unterbringungskosten zusammensetzt) nicht leisten
% können.

% Außerdem würden wir uns wünschen, den Eigenbeitrag für alle
% Teilnehmenden etwas reduzieren zu können -- wir glauben, dass der Preis
% sonst eine Hemmschwelle schafft, die insbesondere solche begabten Kinder
% von einer Teilnahme abhält, die zwar viel Freude an der Mathematik
% empfinden, deren Talent bisher aber nicht spezifisch gefördert wird.

Sehen Sie Möglichkeiten, wie wir finanzielle Förderung gewinnen können? Ich bin Ihnen für jede Hilfe dankbar.

Für Fragen stehe ich Ihnen unter 0821/598-5601 gerne zur Verfügung. Die
Homepage des Matheschülerzirkels finden Sie unter
\textsf{http:/\!/www.math.uni-augsburg.de/schue\-ler/mathezirkel/}.

Mit freundlichen Grüßen

\ \\

Ingo Blechschmidt

\end{document}
