\documentclass[13pt]{zettel}

\renewcommand{\gregor}{\put(9.2,-3.5){\includegraphics[scale=0.18]{campgregor}}}

\usepackage{framed}
\definecolor{shadecolor}{rgb}{.97,.97,.97}
\begin{document}

\renewcommand{\betreff}{Anmeldung zum Mathecamp des Matheschülerzirkels Augsburg}

\makeletterhead{\emph{Bitte in Druckbuchstaben ausfüllen und
bis zum 12. Juli 2014 an obige Adresse oder per Fax an 0821/598-2090 schicken.
Dieses Formular gibt es auch im Internet unter
\href{http://www.math.uni-augsburg.de/schueler/mathezirkel/}{http:/\!/www.math.uni-augsburg.de/schueler/mathezirkel/}.
}}

Hiermit melde ich meine Tochter/meinen Sohn \freistLaenger{} zum Mathecamp des
Matheschülerzirkels Augsburg vom 16. bis 20. August 2014 in
Violau verbindlich an.

\vspace{-0.5em}
\doublespacing
\begin{tabbing}
  Teilnahme am: \= \kill
  Adresse: \> \freistLang \\
  \> \freistLang \\
  E-Mail: \> \freistLang \\
  Telefon: \> \freistLang \\
  Geboren am: \> \freistLang \\
  Schule: \> \freistLang \\
  Klassenstufe: \> \freistKurz
\end{tabbing}

\begin{shaded}
\textbf{Notfallkontakt.} Bei Bedarf sollen benachrichtigt werden (Name, Handy/Telefon):
\begin{enumerate}
\item \freist{7cm},\quad\freist{7cm}
\item \freist{7cm},\quad\freist{7cm}
\end{enumerate}
\end{shaded}

\begin{shaded}
\vspace{-2em}
\begin{tabbing}
  über (Vorname, Nachname): \= \kill
  \textbf{Krankenversicherung.} Mein Kind ist gesetzlich/privat krankenversichert: \\
  über (Vorname, Nachname): \> \freistLang \\
  Krankenkasse: \> \freistLang \\
  Mitgliedsnummer: \> \freistLang
\end{tabbing}
\vspace{-1em}
\end{shaded}

\newpage
\vspace*{-1.5cm}
\enlargethispage{1.0cm}
\small
\singlespacing

\begin{shaded}
\textbf{Anreise.}
\begin{itemize}
\item[\checkbox] Mein Kind kommt am 16. August um 09:30 Uhr zur Universität
Augsburg und fährt dann mit einem vom Mathezirkelteam organisierten Bus nach Violau.
\item[\checkbox] Mein Kind findet sich zwischen 10:00 Uhr und 11:00
Uhr direkt im Bruder-Klaus-Heim in Violau ein. (Eine spätere Anreise ist
gegebenenfalls auch möglich -- bitte uns Bescheid geben.)
\end{itemize}
\end{shaded}

\begin{shaded}
\textbf{Abreise.}
\begin{itemize}
\item[\checkbox] Mein Kind fährt am 20. August mit dem Mathezirkelbus zurück nach Augsburg und darf sich
ab dem Campus der Universität selbstständig auf den Heimweg machen.
\item[\checkbox] Mein Kind fährt mit dem Bus zurück nach Augsburg und wird
pünktlich um 17:30 Uhr auf dem Campus der Universität von mir oder folgender
Person abgeholt:

\vspace{0.3em}
\freistLang

\item[\checkbox] Ich hole mein Kind zwischen 16:30 Uhr und 17:30
Uhr direkt in Violau ab. (Eine spätere Abreise ist gegebenenfalls auch möglich
-- bitte uns Bescheid geben.) Ich berechtige auch folgende Person, mein Kind
abzuholen:

\vspace{0.3em}
\freistLang
\item[\checkbox] Mein Kind darf direkt von Violau aus selbstständig abreisen.
\end{itemize}
\end{shaded}

\begin{shaded}
\textbf{Aktivitäten.}
\begin{itemize}
  \item[\checkbox] Mein Kind darf sich sportlich betätigen.
  \item[\checkbox] Mein Kind darf schwimmen.
  \item[\checkbox] Mein Kind darf (ab achte Klasse) in Gruppen ab drei Schülern
  auch ohne unmittelbare Aufsicht für kurze Zeit das Gelände verlassen.
\end{itemize}
\end{shaded}

\begin{shaded}
\textbf{Sonstiges.}
\begin{itemize}
  \item[\checkbox] Mein Kind nimmt folgende Medikamente: \\[1em] \freistLang
  \item[\checkbox] Mein Kind hat folgende gesundheitliche Beeinträchtigungen
  (etwa Allergien): \\[1em] \freistLang
  \item[\checkbox] Während des Camps entstandene Fotoaufnahmen dürfen
  vom Matheschülerzirkel für die Vereinsarbeit verwendet werden.
  \item[\checkbox] Mein Kind nimmt Bettwäsche mit. (Gegen einen Beitrag
  von~5~\texteuro{} kann Bettwäsche auch vom Heim gestellt werden.)
  \item[\checkbox] Sonstige Hinweise (zum Beispiel vegetarisches Essen): \\[1em]
  \freistLang
\end{itemize}
\end{shaded}

\begin{shaded}
\textbf{Mathematik.} Folgende besonderen Themenwünsche:
\freist{6.2cm}
\end{shaded}

\begin{tabbing}
  \freistMittel \qquad\qquad \= \kill
  \freistMittel \> \freistLaenger \\
  Ort, Datum \> Unterschrift Erziehungsberechtigte(r)
\end{tabbing}

\end{document}
