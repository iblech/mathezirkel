\documentclass{zirkelbrief}
\usepackage{textcomp}

\begin{document}

\renewcommand{\anschrift}{%
      Friedrich Stiftung \\
      Schiffgraben 42 \\
      30175 Hannover \\}
\renewcommand{\datum}{\today}
\renewcommand{\betreff}{Matheschülerzirkel Augsburg}
\renewcommand{\absender}{%
      \textbf{Ingo Blechschmidt} \\
      \ \\
      Lehrstuhl für Algebra und Zahlentheorie \\
      Universitätsstr. 14 \\
      86159 Augsburg \\
      \ \\
      Telefon \> +49 (0) 821 598 -- 5601 \\
      Telefax \> +49 (0) 821 598 -- 2090 \\
      \textsf{blechschmidt@math.uni-augsburg.de} \\}

\makeletterhead

Sehr geehrte Kuratorinnen und Kuratoren der Friedrich Stiftung, \\
sehr geehrte Damen und Herren,

der Matheschülerzirkel Augsburg ist ein Programm zur Förderung
des Interesses und der Begeisterung für Mathematik unter Schülerinnen und
Schülern aller weiterführenden Schulen. Es wird ehrenamtlich durch Doktorandinnen und
Doktoranden sowie Studentinnen und Studenten der Universität Augsburg
organisiert. Anbei finden Sie eine ausführliche Beschreibung unserer Projekte.

In den Sommerferien 2015 werden wir mit 90 Schülerinnen und Schülern zum
zweiten Mal ein siebentägiges mathematisches Feriencamp veranstalten. Um den
benötigten Eigenbeitrag der Teilnehmenden gering zu halten, sind wir auf
finanzielle Fördermittel in Gesamthöhe von etwa 10.000 \texteuro{} angewiesen.

Unter diesem Gesichtspunkt wenden wir uns an Sie. Wir glauben, dass unser
Mathe\-schü\-ler\-zir\-kel dem Förderprofil und der Philosophie der Friedrich
Stiftung entspricht und stellen daher einen Antrag auf Förderung. Auch Teilbeiträge
helfen uns bei unserem Bestreben, allen interessierten Schülerinnen und
Schülern unabhängig von ihrem finanziellen Hintergrund eine Teilnahme zu
ermöglichen.

Auf die Friedrich Stiftung aufmerksam gemacht hat uns Ihr  
ehemaliger Kollege Kaspar Spinner. Wir danken Ihnen für die Zeit, die Sie sich
für unsere Bewerbung nehmen, und freuen uns, wenn Sie den Matheschülerzirkel
in Ihrer Förderung berücksichtigen.

Mit freundlichen Grüßen

\ \\

Ingo Blechschmidt

\vspace{-0.2cm}

{\small (stellvertretend für das gesamte Organisationsteam)}


\vfill
\emph{Anlagen:}
\begin{itemize}
\setlength{\itemsep}{-0.3em}
\item Beschreibung unserer Projekte
\item Artikel in der \emph{Augsburger Allgemeinen}
\item Artikel auf einem Online-Portal
\item Pressemitteilung
\item Poster und Flyer
\end{itemize}

\end{document}
