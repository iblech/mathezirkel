\documentclass[12pt]{zettel}

\usepackage{booktabs}

\usepackage{geometry}
\geometry{tmargin=2cm,bmargin=2cm,lmargin=3cm,rmargin=3cm}

\renewcommand{\gregor}{\put(13.2,-3.0){\includegraphics[scale=0.18]{cover}}}

\usepackage{framed}
\definecolor{shadecolor}{rgb}{.97,.97,.97}

\begin{document}

\renewcommand{\betreff}{}

\makeletterhead{}

\vspace{-2em}

\begin{center}
  \Large\textbf{\textsf{Bewerbung um den Witty-Förderpreis 2014: \\
  Matheschülerzirkel Augsburg }}
\end{center}

Aus der Anforderungsbeschreibung: "`Das Projekt soll
Kindern/Jugendlichen Selbstwertgefühl vermitteln, sie fördern und nachhaltig
Hilfe zur Selbsthilfe bieten."'

\begin{itemize}
\item Information über Verein und Aktivitäten inkl. evtl. Presseberichte
\item Wie setzen sich die rd. 50 TeilnehmerInnen am Mathematikcamp zusammen? Kann
man sich einfach bewerben - läuft es über Lehrer bzw. Schulen?
\item Wie sieht es mit der "`Nachhaltigkeit"' aus? d.h. wie geht es weiter, wenn
einzelne Schüler richtig +Feuer fangen und Begabung zeigen?
\item Wie lauten Ihre Ziele bei diesen Aktivitäten und Projekten? Gibt es zu
wenig Mathematikstudenten und setzen Sie deshalb bereits so früh an?
\item Das Preisgeld beträgt 10.000 Euro. Für das Camp benötigen Sie 7000,--. Wie
würden Sie das restliche Preisgeld sinnvoll verwenden?
\end{itemize}

\section{Bewerber}

Wir sind Doktoranden und Mitarbeiter des Instituts für Mathematik der
Universität Augsburg.

Kurzvorstellung

Zielsetzung


\section{Projektbeschreibung}

Wofür soll das Preisgeld eingesetzt werden?

Was macht das Projekt einmalig/einzigartig für unsere Region?


\section{Vision}

Was soll mit dem Projekt bewirkt werden?


\section{Zielgruppe(n)}

Wer lässt sich durch das Projekt erreichen?

Welchen Anspruch hat es?


\section{Budget}

Wir selbst arbeiten ehrenamtlich. Finanzielle Unterstützung benötigen wir aber
für die Veranstaltungen, die wir durchführen. Die größte Posten nehmen dabei das
Mathecamp (7.000,--~\texteuro) und die Durchführung der XXXten Stufte der
Mathematik-Olympiade ein (2.000,--~\texteuro). Materialien für die
Zirkelarbeit, Preise für Schülerinnen und Schüler und die Durchführung von
Abschluss- und Auftaktveranstaltungen kosten etwa 1.000,--~\texteuro.

Der Universität gilt insofern Dank, als dass wir unentgeltlich ihre
Räumlichkeiten nutzen können und Büromaterialien, Briefporto und ähnliche
Posten über sie abwickeln können.

Nachstehend unsere detaillierte Kalkulation. Möglicherweise übrig bleibende
Mittel können wir sinnvoll im nächsten Jahr verwenden, schließlich sollen das
Mathecamp und die restlichen Veranstaltungen regelmäßig jedes Jahr durchgeführt
werden.

\begin{center}
\renewcommand{\arraystretch}{1.3}
\begin{tabular}{@{}p{5cm}@{\qquad}r@{\qquad}p{6cm}@{}}
  \toprule
  \textbf{Mathecamp} & insges. 6.126 \texteuro \\
  Unterkunft mit Verpflegung & 7.336 \texteuro & 30 \texteuro{} pro Nacht und
  Person zzgl. 11 \texteuro{} Mittagessen am letzten Tag \\
  An- und Abreise & 300 \texteuro & Busunternehmen XXX \\
  Versicherung & 140 \texteuro & 2,50 \texteuro{} pro Person \\
  Sonstiges & 1.500 \texteuro & Workshop-Materialien,
  Zwischenmahlzeiten, Freizeitaktivitäten, Benzinkosten eines Autos vor Ort,
  diverse kleinere Posten \\
  Eigenbeteiligung & $-$3.150 \texteuro & 70 \texteuro{} pro Kind
  (abzüglich etwa 5~Kinder, denen wir die Eigenbeteiligung erlassen) \\\\
  \textbf{Abschlussveranstaltung} & insges. 500 \texteuro &
  Verpflegung und Preise (im Juli) \\\\
  \textbf{Auftaktveranstaltung} & insges. 300 \texteuro &
  Verpflegung (im September) \\\\
  \textbf{Mathematik-Olympiade} & insges. 2.000 \texteuro \\\\
  \textbf{Kursmaterialien} & insges. 1.500 \texteuro &
  Bücher zur Kursvorbereitung,
  Anschauungsmaterialien, XXX \\
  \bottomrule
\end{tabular}
\end{center}

Falls Geld übrig bleibt, kaufen wir noch: XXX


\section{Öffentlichkeitsarbeit}

Um auf die Initiierung unseres Projekts zu Beginn des Schuljahrs~2013/2014 auf
uns aufmerksam zu machen, schickten wir allen Gymnasien Schwabens und einigen
weiteren Schulen im Umkreis von Augsburg Informationspakete mit Lehrerbriefen,
Flyern und Plakaten. Um sicherzugehen, dass unser Angebot in der
Vielzahl der Korrespondenz bei den Schulen nicht unterging, befragten wir außerdem
die Studenten der Universität nach Lehrern, die zu ihrer Schulzeit ein
besonders hohes Ausmaß an Engagement zeigten, und schrieben diese separat an.

Ferner unterstützte uns mit der Öffentlichkeitsarbeit das Kultusministerium,
unter anderem dadurch, indem es separat von unseren Briefen den Aufruf zur
Beteiligung auch noch malXXX an die Schulen weiterleitete.

Schließlich gaben wir eine Pressemitteilung heraus, die von der
Augsburger Allgemeinen aufgegriffen und zu einem großenXXX Artikel aufbereitet
wurde. Als das Projekt angelaufen war, kam ferner das Augsburger
Regionalfernsehen a.tv auf uns zu.

Auf diese Weise konnten wir insgesamt etwa~250 Schülerinnen und Schüler für
unser Projekt begeistern, davon~XXX aus dem Landkreis Augsburg. Um Werbung für
das Mathecamp zu machen, nutzen wir vor allem den bereits etablierten Kontakt
und informieren unsere Schülerinnen und Schüler in den Seminaren persönlich und
zusätzlich per Brief. Ferner verfassen wir wieder eine Pressemitteilung und
informieren die Augsburger Allgemeine.

Selbstverständlich sind wir auch im Internet auf den Seiten der Universität
vertreten (\textsl{http:/\!/www.math.uni-augsburg.de/schueler/mathezirkel/})
und schülerfreundlich über Facebook zu erreichen. Über den
Mathematisch-Physikalischen Verein~e.\,V. erreichen wir Alumni und Freunde der
Universität, die im Bekanntenkreis ebenfalls auf uns aufmerksam machen können.


\section{Zeitrahmen}

Projektstart und voraussichtliche Dauer?

Sind
Verlängerungen
eingeplant?


\section{Ansprechpartner}

Die Hauptorganisatoren sind Ingo Blechschmidt, Kathrin Helmsauer und Sven
Prüfer. Sie erreichen uns telefonisch unter 0821/598-5601, 0821/598-5795 bzw.
0821/598-5805. Eine allgemeine E-Mail-Adresse, die uns alle erreicht, ist
\textsf{mathezirkel@math.uni-augsburg.de}. Unsere persönlichen Adressen sind
\textsf{ingo.blechschmidt@math.uni-augsburg.de},
\textsf{kathrin.helmsauer@math.uni-augsburg.de} bzw.
\textsf{sven.pruefer@math.uni-augsburg.de}. Unsere Post-Adresse lautet:

\begin{tabbing}
  Matheschülerzirkel Augsburg \\
  Lehrstuhl für Algebra und Zahlentheorie \\
  Universitätsstraße 14 \\
  86159 Augsburg
\end{tabbing}


\section{Erfolgskontrolle}

Unmittelbar und rein qualitativ können wir den Erfolg an den Rückmeldungen der
Kinder und ihrer Eltern messen: Hat den Kindern das Camp und allgemeiner der
gesamte Mathezirkel Spaß, Freude und Interesse bereitet? Gibt es
Verbesserungsvorschläge, Wünsche für das Folgejahr oder anderweitige Kritik?

Quantitativer können wir unseren Erfolg anhand der Teilnehmerzahlen im nächsten
Jahr messen: Wenn den Kindern unsere Veranstaltungen gefallen, werden sie sich
nächstes Jahr wieder anmelden und vielleicht sogar Freunde mitbringen.

Langfristig können wir auch verfolgen, wie viele unsere Teilnehmer später ein
Studium in den Bereichen Mathematik, Informatik, Naturwissenschaft und Technik
beginnen. Auf die Steigerung von solchen Studienzahlen legen wir aber kein
besonderes Augenmerk -- andere Fächer sind ja ebenfalls interessant! Wichtig
ist uns, die jetzt vorhandende Begabung und das Interesse zu fördern. Einen Weg
wollen wir nicht aufzeigen. XXX super schlecht

In unserem ersten Jahr erhielten wir auch schon sehr positive Rückmeldungen der
Kinder und Eltern. Bestätigung der Präsenzzirkel erhielten wir insofern, als
dass die Teilnehmerzahlen nur marginal zurückgegangen sind (von etwa 9~Kindern
pro Gruppe auf~7~Kinder).

\end{document}
