\documentclass{serienbrief}

\geometry{tmargin=0.5cm,bmargin=1.5cm,lmargin=1.8cm,rmargin=2.7cm}

\def\chopline#1;#2;#3;#4;#5 \\{
  \def\schule{#1}
  \def\strasse{#2}
	\def\plz{#3}
	\def\ort{#4}
 }
 
 \newif\ifmore \moretrue


\begin{document}

\newread\quelle
 \openin\quelle=adressen.dat
 
 \loop
  \read\quelle to \zeile
  \ifeof\quelle
   \global\morefalse
  \else
   \expandafter\chopline\zeile\\

\renewcommand{\anschrift}{
      \schule \\
      Fachbereich Mathematik \\
      \strasse \\
      \plz ~\ort
			}

\renewcommand{\datum}{\today}
\renewcommand{\betreff}{Mathesch\"ulerzirkel der Universit\"at Augsburg}


\makeletterhead

Sehr geehrte Lehrerin, sehr geehrter Lehrer,

beginnend mit dem Schuljahr 2013/2014 bietet die Universit\"at Augsburg
einen \emph{Mathe\-matik\-zir\-kel} f\"ur interessierte Sch\"ulerinnen und Sch\"uler der
Jahrgangsstufen~5 bis~12 an.

Dabei sollen bei regelm\"a\ss ig stattfindenden Treffen, \emph{Pr\"asenzzirkeln}, in
den R\"aumlichkeiten der Universit\"at den Sch\"ulern spannende mathematische Themen
n\"ahergebracht werden; es geht nicht um Nachhilfe oder Vorwegnahme von
Unterrichtsinhalten, sondern um Besch\"aftigung mit anderen Themen wie etwa
Spieltheorie, Zahlenr\"atseln, Fraktalen oder Graphentheorie. Falls es gew\"unscht
wird, ist auch eine Vorbereitung auf Wettbewerbe wie den Landes- und
Bundeswettbewerb oder den K\"anguru der Mathematik m\"oglich.

%F\"ur diejenigen Sch\"uler, die an den Pr\"asenzzirkeln nicht teilnehmen k\"onnen, gibt
%es schriftliche \emph{Korrespondenzzirkel}.
Zus\"atzlich bieten wir schriftliche \emph{Korrespondenzzirkel} an, an denen auch diejenigen Sch\"uler teilnehmen k\"onnen, die zu den Pr\"asenzzirkeln verhindert sind. Dabei erhalten die Sch\"uler von uns
Materialien und bearbeiten \"Ubungsbl\"atter, welche wir dann korrigiert zur\"uckgeben.

Die Kurse werden von Mitarbeiterinnen und Mitarbeitern des Instituts f\"ur
Mathematik gehalten, die zum Teil schon selbst von derartigen Programmen
profitiert haben. Die Teilnahme ist nat\"urlich f\"ur die Sch\"uler kostenlos. Des Weiteren finden die Pr\"asenzzirkel
au\ss erhalb der Unterrichtszeiten statt.

F\"ur alle Sch\"uler, die sich f\"ur eine Teilnahme an einem Mathematikzirkel
interessieren, findet eine Er\"offnungsveranstaltung am 9.11.2013 um 10:00 Uhr mit einem Vortrag von
\linebreak
\vspace{-2.2em}
\begin{center}
    Prof. Dr. Jost-Hinrich Eschenburg\\
    \emph{Was sind eigentlich die Zahlen?}
\end{center}
\vspace{-1em}
im H\"orsaal 1004 des Physikh\"orsaalzentrums der Universit\"at statt, eine
Anmeldung ist nicht erforderlich. Die konkreten Termine f\"ur die
ersten Pr\"asenzzirkel und der Ablauf der Korrespondenzzirkel werden auch dort
abgesprochen.

Ich bitte Sie, Ihre Sch\"uler auf unser Angebot aufmerksam zu machen, sie zur
Teilnahme zu ermuntern und geeignete Sch\"uler gezielt
anzusprechen. \emph{Die einzigen Teilnahmevoraussetzungen sind Spa\ss\ und Interesse an der Mathematik.} Selbstverst\"andlich k\"onnen wir Ihnen weitere Flyer
zukommen lassen. F\"ur R\"uckfragen stehe ich Ihnen jederzeit zur Verf\"ugung. Wenn gew\"unscht, kommen wir auch
gerne an Ihre Schule, um Ihren Sch\"ulern eine Kostprobe zu geben. Vielen Dank f\"ur Ihre
Unterst\"utzung.

Mit freundlichen Gr\"u\ss en
\vspace{0.0cm}

\hspace{1cm} \includegraphics[scale=0.4]{unterschrift_marco_hien}

\vspace{-0.4cm}

Prof. Dr. Marco Hien

\vspace{-0.2cm}

{\small Studiendekan der Mathematik}

\newpage

 \fi
 \ifmore\repeat
 
 \closein\quelle


\end{document}
