\documentclass{zirkelbrief}

\begin{document}

\renewcommand{\anschrift}{%
      Holbein-Gymnasium Augsburg \\
      Fachbereich Mathematik \\
      Hallstraße 10 \\
      86150 Augsburg}
\renewcommand{\datum}{2.9.2013}
\renewcommand{\betreff}{Matheschülerzirkel der Universität Augsburg}

\makeletterhead

Sehr geehrte Frau Mathespaß,

beginnend mit dem Schuljahr 2013/2014 bietet die Universität Augsburg
einen \emph{Mathematikzirkel} für mathematisch interessierte Schülerinnen und Schüler der
Jahrgangsstufen~5 bis~12 an.

Dabei sollen bei regelmäßig stattfindenden Treffen, \emph{Präsenzzirkeln}, in
den Räum\-lich\-kei\-ten der Universität spannende mathematische Themen den Schülern
näher gebracht werden; es geht nicht um Nachhilfe oder Vorwegnahme von
Unterrichtsinhalten, sondern um Beschäftigung mit orthogonalen Themen wie etwa
Spieltheorie, Zahlenrätsel, Fraktale oder Graphentheorie. Falls gewünscht, ist
auch eine Vorbereitung auf mathematische Wettbewerbe möglich.

Für diejenigen Schüler, die an den Präsenzzirkeln nicht teilnehmen können, gibt
es schriftliche \emph{Korrespondenzzirkel}. Dabei bearbeiten die Schüler
Übungsblätter, welche wir dann korrigieren.

Die Kurse werden von Mitarbeiterinnen und Mitarbeitern des
Instituts für Mathematik gehalten. Die Kosten übernimmt die Universität, sodass
die Teilnahme für die Schüler kostenlos ist. Die Präsenzzirkel finden natürlich
außerhalb der Schulzeiten statt, die genauen Termine werden in einer
Auftaktveranstaltung abgesprochen.

Wir bitten Sie, die beigelegten Briefe mit ausführlicheren Informationen und Flyer an Ihre
Fachlehrerinnen und Fachlehrer weiterzuleiten und Ihre Schüler auf unser
Angebot aufmerksam zu machen. Wenn gewünscht, kommen wir auch gerne zu Ihnen an die
Schule, um Ihnen und Ihren Kolleginnen und Kollegen oder Ihren Schülern das
Angebot persönlich zu erläutern.

Für Rückfragen stehen wir Ihnen jederzeit telefonisch und schriftlich zur
Verfügung. Vielen Dank für Ihre Unterstützung.

Mit freundlichen Grüßen,

\end{document}

Zu diskutieren:
* Wettbewerbsbetonung?
* Beispielthemen, etwa Nim-Spiele und Graphentheorie
* Terminoptimismus/-pessimismus
* doch "profitiert haben"?
* "Einzige Bedingung für die Teilnahme ist Spaß und Interesse an der Mathematik."
* Mamazettel (klar, für die Variante an die Schüler)
* "Ich" oder "wir"? Schließlich unterschreibt nur einer den Brief!
* "Universität" vs. "Universität Augsburg"? Unterscheidung je nach Empfänger?
* Im Brief an Lehrer/Schüler: Anmeldeformalitäten
* In der Fassung an die Fachbetreuer sollte rein, dass sie bei uns ggf. Flyer
  nachbestellen können.
* 5. Klasse separat behandeln? Die müssen sich ja erst orientieren.
* Kathrin: Ausserdem: Die Praesenzzirkel sind fuer alle, oder?
  Ingo: Was meinst du damit?
* Wie von Kathrin beobachtet, passt der Satz "Die Präsenzzirkel werden
  natürlich außerhalb der Schulzeiten stattfinden." nicht gut in den Fluss.

\end{document}
