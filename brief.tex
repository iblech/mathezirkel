\documentclass[a4paper,ngerman,12pt]{scrartcl}

\usepackage[utf8]{inputenc}

\usepackage[ngerman]{babel}

\usepackage{amsmath,amsthm,amssymb,stmaryrd,color,graphicx}

\usepackage[protrusion=true,expansion=true]{microtype}

\usepackage{lmodern}

\usepackage{hyperref}

\setlength\parskip{\medskipamount}
\setlength\parindent{0pt}

\clubpenalty=10000
\widowpenalty=10000
\displaywidowpenalty=10000

\pagestyle{empty}

\begin{document}

\setlength{\unitlength}{1cm}
\begin{picture}(0,0)
  \put(0,0){%
    {\tiny Universität Augsburg -- Lst. Algebra und Zahlentheorie -- 86159
    Augsburg}
  }
  \put(0,-3){\vbox{
    Peter Mustermann \\
    Beispielstraße 47 \\
    80815 Irgendwo
  }}
  \put(10,-5){\vbox{
    \begin{tabbing}
      Telefax \= \kill
      \textbf{Prof. Dr. Marco Hien} \\
      \ \\
      \small
      Lehrstuhl für Algebra und Zahlentheorie \\
      \small
      Universitätsstr. 14 \\
      \small
      86159 Augsburg \\
      \ \\
      \small
      Telefon \> +49 (0) 821 598 -- 2152 \\
      \small
      Telefax \> +49 (0) 821 598 -- 2090 \\
      \small
      \texttt{marco.hien@math.uni-augsburg.de} \\
      \\
      \small
      \url{http://www.math.uni-augsburg.de/alg}
    \end{tabbing}
    \rule{6cm}{0.25mm} \\
    \small 
    Augsburg, den 2.9.2013
  }}
\end{picture}

\vspace{5cm}

\textbf{Matheschülerzirkel der Universität Augsburg}

Sehr geehrte Damen und Herren,

beginnend mit dem Schuljahr 2013/2014 bietet die Universität Augsburg
Mathematikzirkel für mathematisch interessierte Schülerinnen und Schüler der
Jahrgangsstufen 5 bis 12 an.

Dabei sollen bei regelmäßig stattfindenden Treffen, sog. Präsenzzirkeln, in den
Räum\-lich\-kei\-ten der Universität interessante mathematische Themen den
Schülerinnen und Schülern näher gebracht werden. Für diejenigen Schüler, die an
den Präsenzzirkeln nicht teilnehmen können, wird es schriftliche
Korrespondenzzirkel geben. Insgesamt geht es in diesen Zirkeln nicht um
Nachhilfe oder Vorwegnahme von Unterrichtsinhalten, sondern um Beschäftigung
mit weiterführenden und tiefergehenden Themen (wie etwa Nim-Spiele und
Graphentheorie). Falls es gewünscht wird, ist auch eine Vorbereitung auf
mathematische Wettbewerbe (zum Beispiel Landes- und Bundeswettbewerb
Mathematik, Mathematikolympiade und Känguru-Wettbewerb) möglich. 

Die Kurse werden von Mitarbeiterinnen und Mitarbeiter des Instituts für
Mathematik gehalten, die zum Teil schon selber von derartigen Programmen
profitierten und über langjährige Erfahrung in dieser Art von Lehre besitzen.
Die Kosten übernimmt die Universität, sodass die Teilnahme für die Schüler
kostenlos ist. Die Präsenzzirkel werden natürlich außerhalb der Schulzeiten
stattfinden, die konkreten Termine für die ersten Präsenzzirkel und der Ablauf
der Korrespondenzzirkel werden in einer ersten Eröffnungsveranstaltung
abgesprochen.

Wir bitten Sie, die beigelegten Briefe und Informationsmaterialien an Ihre
Fachlehrerinnen und Fachlehrer weiterzuleiten und Ihre Schüler auf unser
Angebot aufmerksam zu machen. Wenn gewünscht, kommen wir auch gerne zu Ihnen an
die Schule, um Ihnen und Ihren Kolleginnen und Kollegen oder Ihren Schülern das
Angebot persönlich zu erläutern. 

Für Rückfragen stehen wir Ihnen jederzeit Verfügung. Vielen Dank für Ihre
Unterstützung. 

Mit freundlichen Grüßen,


\end{document}
