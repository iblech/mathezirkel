\documentclass{zirkelbrief}

\graphicspath{{../../illustrationen/}}

\geometry{tmargin=0.5cm,bmargin=1.5cm,lmargin=1.8cm,rmargin=2.7cm}

\begin{document}

\renewcommand{\anschrift}{%
      Holbein-Gymnasium Augsburg \\
      Fachbereich Mathematik \\
      Hallstraße 10 \\
      86150 Augsburg}
\renewcommand{\datum}{\today}
\renewcommand{\betreff}{Matheschülerzirkel der Universität Augsburg}

\makeletterhead

Sehr geehrte Lehrerin, sehr geehrter Lehrer,

der Mathezirkel ist ein kostenloses Förderangebot der Universität Augsburg für interessierte SchülerInnen
der Klassenstufen~5 bis~12. Nach unserer Gründung im letzten Schuljahr geht es nun in die nächste Runde.

Wir bringen den SchülerInnen bei regelmäßig stattfindenden Treffen in
den Räumlichkeiten der Universität, \emph{Präsenzzirkeln}, spannende mathematische Themen
näher. Dabei geht es nicht um Nachhilfe oder Vorwegnahme von
Unterrichtsinhalten, sondern um Beschäftigung mit anderen Themen wie etwa
Spieltheorie, Zahlenrätseln, Fraktalen oder Graphentheorie. Falls es gewünscht
wird, ist auch eine Vorbereitung auf Wettbewerbe wie den Landes- und
Bundeswettbewerb oder den Känguru der Mathematik möglich.

%Für diejenigen Schüler, die an den Präsenzzirkeln nicht teilnehmen können, gibt
%es schriftliche \emph{Korrespondenzzirkel}.
Zusätzlich bieten wir schriftliche \emph{Korrespondenzzirkel} an, an denen auch diejenigen Schüler teilnehmen können, die zu den Präsenzzirkeln verhindert sind. Dabei erhalten die Schüler von uns
Materialien und bearbeiten Übungsblätter, welche wir dann korrigiert zurückgeben.

Die Kurse werden von Mitarbeiterinnen und Mitarbeitern des Instituts für
Mathematik gehalten, die zum Teil schon selbst von derartigen Programmen
profitiert haben. Selbstverständlich finden die Präsenzzirkel
außerhalb der Unterrichtszeiten statt.
Im letzten Schuljahr nahmen etwa 250 SchülerInnen unsere
Angebote wahr, davon kamen knapp die Hälfte aus dem Großraum Augsburg. Außerdem veranstalteten wir mit knapp 90~Kindern ein fünftägiges Mathecamp im Schullandheim Violau.

Für alle SchülerInnen, die sich für eine Teilnahme an einem Mathematikzirkel
interessieren, findet am 18.10.2014 eine Eröffnungsveranstaltung mit einem Vortrag von
Prof. Dr. Kai Cieliebak zum Thema \emph{Unberechenbare Zahlen}
statt, eine
Anmeldung ist nicht erforderlich. Dort sprechen wir auch die konkreten Termine für die
ersten Präsenzzirkel und den Ablauf der Korrespondenzzirkel ab.

Ich bitte Sie, Ihre SchülerInnen auf unser Angebot aufmerksam zu machen, sie zur
Teilnahme zu ermuntern und geeignete SchülerInnen gezielt
anzusprechen. \emph{Einzige Teilnahmevoraussetzungen sind Spaß und
Interesse an der Mathematik.} SchülerInnen können gerne bei uns
hineinschnuppern und bei Nichtgefallen ohne bürokratischen Aufwand wieder
aufhören.

Für Rückfragen stehe ich Ihnen jederzeit zur Verfügung. Wenn gewünscht, kommen wir auch
gerne an Ihre Schule, um Ihren SchülerInnen eine Kostprobe zu geben. Ich danke
Ihnen vielmals für Ihre Unterstützung.

Mit freundlichen Grüßen
\vspace{0.0cm}

\hspace{1cm} \includegraphics[scale=0.4]{unterschrift_marco_hien}

\vspace{-0.4cm}

Prof. Dr. Marco Hien

\vspace{-0.2cm}

{\small Studiendekan der Mathematik}

\end{document}
