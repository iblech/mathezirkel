\documentclass{zirkelbrief}

\graphicspath{{../../illustrationen/}}

\begin{document}

\renewcommand{\anschrift}{%
      Holbein-Gymnasium Augsburg \\
      Fachbereich Mathematik \\
      Hallstraße 10 \\
      86150 Augsburg}
\renewcommand{\datum}{22.9.2014}
\renewcommand{\betreff}{Matheschülerzirkel der Universität Augsburg}

\makeletterhead

Sehr geehrte Frau Mathespaß,

der Matheschülerzirkel ist ein kostenloses Förderangebot der Universität Augsburg für interessierte SchülerInnen
der Klassenstufen~5 bis~12. Nach unserer Gründung im letzten Schuljahr geht es nun in die nächste Runde.

In regelmäßig stattfindenden Treffen in den Räum\-lich\-kei\-ten der Universität, \emph{Präsenzzirkeln}, bringen wir
den SchülerInnen spannende mathematische Themen
näher. Es geht nicht um Nachhilfe oder Vorwegnahme von
Unterrichtsinhalten, sondern um Beschäftigung mit anderen Themen wie etwa
Spieltheorie, Zahlenrätseln, Fraktalen oder Graphentheorie. Falls gewünscht,
bereiten wir auch auf mathematische Wettbewerbe vor.

Zusätzlich bieten wir schriftliche \emph{Korrespondenzzirkel} an, an denen auch diejenigen Schü\-ler\-In\-nen teilnehmen können, die zu den Präsenzzirkeln verhindert sind. Dabei erhalten die SchülerInnen von uns
Materialien und bearbeiten Übungsblätter, welche wir dann korrigieren.

Die Kurse werden von Mitarbeiterinnen und Mitarbeitern des
Instituts für Mathematik gehalten. Die Teilnahme ist für die SchülerInnen kostenlos. Die Präsenzzirkel finden
außerhalb der Unterrichtszeiten statt, wobei die genauen Termine in einer
Auftaktveranstaltung abgesprochen werden.

Im letzten Schuljahr nahmen etwa 250 SchülerInnen unsere
Angebote wahr, davon kamen knapp die Hälfte aus dem Großraum Augsburg. Außerdem veranstalteten wir mit knapp 90~Kindern ein fünftägiges Mathecamp im Schullandheim Violau.

Ich bitte Sie, die beigelegten Briefe mit ausführlicheren Informationen und Flyern an Ihre
FachlehrerInnen weiterzuleiten und Ihre SchülerInnen auf unser
Angebot aufmerksam zu machen. Wenn gewünscht, kommen wir auch gerne zu Ihnen an die
Schule, um Ihnen und Ihren KollegInnen oder Ihren SchülerInnen das
Angebot persönlich zu erläutern.

Für Rückfragen stehe ich Ihnen jederzeit telefonisch und schriftlich
zur Verfügung. Ich danke Ihnen vielmals für Ihre Unterstützung.

Mit freundlichen Grüßen

\vspace{1cm}

Prof. Dr. Marco Hien

\vspace{-0.2cm}

{\small Studiendekan der Mathematik}

\end{document}
