\documentclass{../zirkelbrief}

\begin{document}

\renewcommand{\anschrift}{%
      ((schule)) \\
      ((lehrername)) \\
      ((strasse)) \\
      ((plz)) ((ort))}
\renewcommand{\datum}{\today}
\renewcommand{\betreff}{Matheschülerzirkel der Universität Augsburg}
\renewcommand{\absender}{%
      \textbf{Ingo Blechschmidt} \\
      \ \\
      Lehrstuhl für Algebra und Zahlentheorie \\
      Universitätsstr. 14 \\
      86159 Augsburg \\
      \ \\
      Telefon \> +49 (0) 821 598 -- 5601 \\
      Telefax \> +49 (0) 821 598 -- 2090 \\
      \textsf{blechschmidt@math.uni-augsburg.de} \\}

\makeletterhead

\newif\iflehrermaennlich\lehrermaennlich((lehrermaennlich))
\newif\ifschuelermaennlich\schuelermaennlich((schuelermaennlich))
\newif\ifweitweg\weitweg((weitweg))

\iflehrermaennlich
Sehr geehrter ((lehrername)),
\else
Sehr geehrte ((lehrername)),
\fi

ich bin ein Doktorand in Mathematik an der Universität Augsburg. Vor einem Jahr
schrieb ich Sie an, um auf den damals neu initiierten Matheschülerzirkel der
Universität aufmerksam zu machen. Nun geht unser Projekt in die nächste Runde.

Wie im letzten Jahr richten wir uns an alle SchülerInnen Schwabens der
Klassenstufen~5 bis~12, die Spaß und Interesse an Mathematik haben. Wir bieten
regelmäßig stattfindende \emph{Präsenzzirkel}, bei denen wir auf dem Campus der
Universität in kleinen Gruppen spannende Mathematik machen.
\ifweitweg
Für weiter entfernt wohnende SchülerInnen gibt es schriftliche
\emph{Korrespondenzzirkel}, bei denen die SchülerInnen von uns Materialien
erhalten und Übungsblätter bearbeiten, welche wir dann korrigiert zurückgeben.
\else
Für SchülerInnen, die zu den Treffen verhindert sind, gibt es
schriftliche \emph{Korrespondenzzirkel}, bei denen die SchülerInnen von uns Materialien
erhalten und Übungsblätter bearbeiten, welche wir dann korrigiert zurückgeben.
\fi
Außerdem veranstalten wir immer in den Sommerferien ein Mathecamp in einem
Schullandheim.
\ifweitweg
Im letzten Schuljahr stammten etwa 130 unserer TeilnehmerInnen nicht aus dem
Großraum Augsburg.
\fi

Die Themen, die wir behandeln, liegen dabei stets abseits des
Schulunterrichts. Wir diskutierten etwa Verschlüsselungssysteme,
Fibonacci-Zahlen, Kettenbrüche, Nim-Spiele, Zahlensysteme, Graphentheorie,
Zahlentheorie, Zauberwürfel, Invarianten, Fraktale, Chaos,
vierdimensionale Geometrie, elliptische Kurven, nichtklassische Logik,
synthetische Differentialgeometrie und Numerik partieller
Differentialgleichungen.

In den nächsten Tagen verschicken wir an alle Gymnasien Schwabens und an ein
paar weitere Schulen Informationsmaterialien und Flyer. Wie letztes Jahr
freue ich mich, wenn Sie am ((schulekurz)) dafür Sorge tragen, dass unser Angebot
an Ihre SchülerInnen weitergegeben wird. Sollte unsere Sendung in der Menge der
Korrespondenz untergehen und Sie nichts erhalten, können Sie mich jederzeit
kontaktieren. Dann schicke ich Ihnen neue Flyer zu.

Ich entschuldige mich für die Verletzung des Dienstwegs und hoffe, dass sich
Ihre KollegInnen nicht unberücksichtigt fühlen. Den Tipp, mich speziell an Sie
zu wenden, gab
\ifschuelermaennlich
Ihr ehemaliger Schüler ((schuelername)).
\else
Ihre ehemalige Schülerin ((schuelername)).
\fi

Vielen Dank für Ihre Unterstützung!

Mit freundlichen Grüßen

\ \\
\ \\

Ingo Blechschmidt

\vspace{-0.2cm}

{\small (stellvertretend für das gesamte Organisationsteam)}

\end{document}
