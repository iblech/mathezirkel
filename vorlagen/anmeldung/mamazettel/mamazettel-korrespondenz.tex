\documentclass{mamazettel}

\begin{document}

\renewcommand{\betreff}{Anmeldung zum Matheschülerzirkel der Universität Augsburg}

\makeletterhead

Meine Tochter/mein Sohn \freist{9.5cm} darf an dem schriftlichen Korrespondenzzirkel
teilnehmen. Hierbei schicken wir Ihrem Kind alle vier Wochen Materialien zu.
Ihr Kind kann dann Übungsblätter bearbeiten und erhält von uns Rückmeldung. Ihre Tochter/Ihr Sohn geht dabei keinerlei Verpflichtung ein.

\doublespacing

\begin{tabbing}
  Teilnahme am: \= entweder \= \kill
  Adresse: \> \freistLang \\
  \> \freistLang \\
  E-Mail: \> \freistLang \\
  Klassenstufe: \> \freistKurz
\end{tabbing}

\begin{tabbing}
  \freistMittel \qquad\qquad \= \kill
  \freistMittel \> \freistLaenger \\
  Ort, Datum \> Unterschrift Erziehungsberechtigte(r)
\end{tabbing}

\vspace{-1.2em}
\scriptsize
Weitere Informationen und Kontaktdaten:
\textsl{http:/\!/www.math.uni-augsburg.de/schueler/mathezirkel}

\small
\onehalfspacing

\vfill
\textbf{Noch drei Fragen\ldots}
\vspace{-1.3em}
\begin{tabbing}
  1. \= Entweder \= \kill
  1. \> \emph{An welcher Art von Korrespondenzzirkel möchtest du teilnehmen?} \\
  \> Entweder \> \checkbox Matheschülerzirkel \\
  \> oder \> \checkbox
  Wettbewerbsvorbereitungszirkel (speziell zum Training auf mathematische \\
  \> \> \phantomcheckbox Wettbewerbe). \\
  \> Bitte nur eine der beiden Varianten ankreuzen. \\
  2. \> \emph{Hast du schon mal an Wettbewerben teilgenommen?} \\
  \> \checkbox Nein. \quad \checkbox Ja, und zwar: \freist{10.8cm} \\
  3. \> \emph{Hast du besondere Themenwünsche oder -vorschläge?} \\[0.5em]
  \> \freist{16cm}
\end{tabbing}

\end{document}
