\documentclass{../../zirkelbrief}

\begin{document}

\renewcommand{\anschrift}{%
      ((schule)) \\
      ((lehrername)) \\
      ((strasse)) \\
      ((plz)) ((ort))}
\renewcommand{\datum}{\today}
\renewcommand{\betreff}{Matheschülerzirkel der Universität Augsburg}
\renewcommand{\absender}{%
      \textbf{Ingo Blechschmidt} \\
      \ \\
      Lehrstuhl für Algebra und Zahlentheorie \\
      Universitätsstr. 14 \\
      86159 Augsburg \\
      \ \\
      Telefon \> +49 (0) 821 598 -- 5601 \\
      Telefax \> +49 (0) 821 598 -- 2090 \\
      \textsf{blechschmidt@math.uni-augsburg.de} \\}

\makeletterhead

\newif\iflehrermaennlich\lehrermaennlich((lehrermaennlich))
\newif\ifschuelermaennlich\schuelermaennlich((schuelermaennlich))
\newif\ifweitweg\weitweg((weitweg))

\iflehrermaennlich
Sehr geehrter ((lehrername)),
\else
Sehr geehrte ((lehrername)),
\fi

ich bin ein Doktorand in Mathematik an der Universität Augsburg und schreibe
Ihnen, da ich Sie auf ein neu initiiertes Projekt der Universität aufmerksam machen
möchte: Ab dem Schuljahr~2013/2014 bieten wir einen Mathematikzirkel für
interessierte Schülerinnen und Schüler der Jahrgangsstufen~5 bis~12 an. Dabei
wird es in den Räumlichkeiten der Universität Augsburg regelmäßig stattfindende
Treffen geben, bei denen wir spannende Themen der Mathematik abseits des
Schulunterrichts behandeln wollen;
\ifweitweg
da ((ort)) ja recht weit von Augsburg entfernt liegt, bieten wir außerdem
schriftliche Korrespondenzzirkel an, bei denen die Schüler von uns Materialien
erhalten und Übungsblätter bearbeiten, welche wir dann korrigiert zurückgeben.
\else
für Schüler, die zu den Treffen verhindert sind, bieten wir außerdem
schriftliche Korrespondenzzirkel an, bei denen die Schüler von uns Materialien
erhalten und Übungsblätter bearbeiten, welche wir dann korrigiert zurückgeben.
\fi

Diesbezüglich verschicken wir in den nächsten Tagen an alle Gymnasien Schwabens
und an ein paar weitere Schulen entsprechende Informationsmaterialien. Nun ist es
so, dass wir von Projekten dieser Art an anderen Orten erfahren haben, dass
solche Briefe in der Menge der Korrespondenz leider manchmal untergehen.
Deshalb erhalten für das ((schulekurz)) speziell Sie diesen Brief vorab:

Wir haben unsere Studenten um Mithilfe gebeten und sie gefragt, ob sie an ihrer
ehemaligen Schule jeweils eine besonders engagierte Mathematik-Lehrerin oder einen besonders
engagierten Mathematik-Lehrer kennen, die bzw. der
vielleicht unser Projekt unterstützen würde. Für das ((schulekurz)) hat sich
\ifschuelermaennlich Ihr ehemaliger Schüler ((schuelername)) \else
Ihre ehemalige Schülerin ((schuelername)) \fi
gemeldet: ((schuelerkurz)) hat Sie in bester
Erinnerung und dachte, dass Sie \iflehrermaennlich der beste Kandidat wären,
\else die beste Kandidatin wären, \fi um an Ihrer Schule dafür Sorge zu tragen,
dass unser Anliegen an die Schüler weitergegeben wird. Darum bitte ich Sie also
ganz herzlich.

Außerdem hoffe ich, dass Sie mir diese Verletzung des Dienstwegs verzeihen, und
dass sich kein anderer Lehrer unberücksichtigt fühlt. Unsere Studentenerhebung
hatte notgedrungen großen Stichprobencharakter.

Vielen Dank für Ihre Unterstützung!

Mit freundlichen Grüßen

\ \\
\ \\

Ingo Blechschmidt

\vspace{-0.2cm}

{\small (stellvertretend für das gesamte Organisationsteam)}

\end{document}
